\chapter*{Abstract}

Do not have an abstract yet.
% Architectural design practice has radically evolved over the course of its history, due to technological improvements that gave rise to advanced automated tools for many design tasks. Traditional paper drawings and scale models are now accompanied by 2D and 3D Computer-Aided Architectural Design (CAAD) software.

% While such tools improved in many ways, including performance and accuracy improvements, the modalities of user interaction have mostly remained the same, with 2D interfaces displayed on 2D screens. The maturation of Augmented Reality (AR) and Virtual Reality (VR) technology has led to some level of integration of these immersive technologies into architectural practice, but mostly limited to visualisation purposes, e.g. to show a finished project to a potential client.

% We posit that there is potential to employ such technologies earlier in the architectural design process and therefore explore that possibility with a focus on Algorithmic Design (AD), a CAAD paradigm that relies on (often visual) algorithms to generate geometries. The main goal of this dissertation is to demonstrate that AR and VR can be adopted for AD activities. 

% To verify that claim, we follow an iterative prototype-based methodology to develop research prototype software tools and evaluate them. The three developed prototypes provide evidence that integrating immersive technologies into the AD toolset provides opportunities for architects to improve their workflow and to better present their creations to clients. Based on our contributions and the feedback we gathered from architectural students and other researchers that evaluated the developed prototypes, we additionally provide insights as to future perspectives in the field.