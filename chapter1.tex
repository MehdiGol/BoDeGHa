\chapter{Introduction}
\epigraph{
``I am going to finish it as soon as possible.'' }{Mehdi Golzadeh}

I started my PhD studies after 10 years of experience in software development. While working as a software developer and in my previous studies I have learned and used several different methodologies, technologies, design patterns, and tools related to software development. I have started with classical software development and also collaborated with my other colleagues in software development using advanced version control tools. I had also gained experience using GitHub as a coding platform.

Today GitHub is the biggest platform for developers to develop open-source and closed-source software and an excellent opportunity to empirically study software development practices. Using Github to learn new things to improve the software development process is a brand new field of study made possible through its abundance of data. My background in software development and my data analytics experience led me to combine both to study trends and defects in GitHub software development. 

In the context of the seco-assist project, I was hired to work in WP4, which was focused on conducting social studies in software development.
I was supervised by Prof. Tom Mens and co-advised by Dr. Alexandre Decan in this work package.
I started with the social aspect of the software development process and discussions of developers on GitHub and later after consulting with my advisors, I decided to study automation tools in GitHub which is also related to the social aspect in a lot of ways that we will discuss later in my dissertation. 
In this document, I present my work on the topic, which I conducted over the last four years, and I aim to explain my methodology as well as my perspective on the subject.

\section{Context}

Software development includes several steps and iterations including collecting requirements, creating applications, testing them, deploying, and maintaining the product. A piece of software goes through all these processes from Idea to launch. These steps are collectively called the software development lifecycle (SDLC).
A step typically produces an output (e.g. an idea, a document, a diagram, or a working piece of software), which functions as an input for the next step. 
Despite their sequential nature, these steps do not end once the software has been delivered to stakeholders, and the software development team repeats them many times to enhance the product.

With the advent of agile development methodologies which break tasks into small modules that are done with minimal planning, iterations are done much more frequently than before and in a shorter period of time.
A full software development cycle is followed during each iteration, including planning, requirements analysis, design, coding and testing which minimizes overall risk and allows the project to adapt to changes quickly.
This results in more work for the developers to perform in each iteration.
% The agile methodology includes several other lightweight methods like Scrum, Extreme Programming (XP), Adaptive Software Development (ASD), Feature Driven Development (FDD), and Dynamic Systems Development Method (DSDM),  Lean software development, etc~\cite{Davis1988}. 



\section{Collaborative Software Development}
\label{sec:csd}

\subsection{Social coding platforms features}

\section{Automation tools}


\subsection{Development bots}
\subsection{CI tools and services}


\section{Thesis statement}

\begin{center}
	\begin{tcolorbox}[sharp corners, colback=blue!10, colframe=gray!80!blue, coltitle=white,fonttitle=\sffamily\large, title=Thesis statement, width=.97\textwidth]
		Bots and CIs are part of software development which can bring automation in collaborative software development and Better technniques are needed to understand and mitigate the problems introduced by interaction of bots in social coding platforms. 
		
	\end{tcolorbox}
\end{center}

\section{Research Questions}
\label{sec:rqa}


\section{Thesis structure}
\label{sec:structure}
